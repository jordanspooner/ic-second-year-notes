\documentclass[twocolumn,english]{article}
\usepackage[latin9]{inputenc}
\usepackage[landscape]{geometry}
\geometry{verbose,tmargin=0.5in,bmargin=0.75in,lmargin=0.5in,rmargin=0.5in}
\setlength{\parskip}{0bp}
\setlength{\parindent}{0pt}
\usepackage{amsmath}

\makeatletter



\usepackage{array}
\usepackage{multirow}
\usepackage{amsbsy}




\providecommand{\tabularnewline}{\\}

\setlength{\columnsep}{0.25in}
\usepackage{xcolor}
\usepackage{textcomp}
\usepackage{listings}
\lstset{
  language=haskell,
  tabsize=2,
  basicstyle=\small\ttfamily,
}



\usepackage{babel}
\usepackage{listings}
\renewcommand{\lstlistingname}{Listing}

\makeatother

\usepackage{babel}
\begin{document}

\title{Reference Sheet for C240 Models of Computation}

\date{Autumn 2017}
\maketitle

\section{Operational Semantics}

\subsection{Simple Expressions}

$E\in\text{SimpleExp}::=n\lvert E+E\lvert E\times E\lvert\dots$

\paragraph{Big-step (Natural)}
\begin{itemize}
\item {\scriptsize{}(B-NUM)} $\begin{array}{c}
\\
\hline n\Downarrow n
\end{array}$.
\item {\scriptsize{}(B-ADD)} $\begin{array}{c}
E_{1}\Downarrow n_{1}\;E_{2}\Downarrow n_{2}\\
\hline E_{1}+E_{2}\Downarrow n_{3}
\end{array}$ (where $n_{3}=n_{1}+n_{2}$).
\end{itemize}

\subparagraph{Properties:}
\begin{itemize}
\item \emph{Determinacy}: For all $E$, $n_{1}$, $n_{2}$, if $E\Downarrow n_{1}$
and $E\Downarrow n_{2}$ then $n_{1}=n_{2}$.
\item \emph{Totality}: For all $E$, there exists an $n$ s.t. $E\Downarrow n$.
\end{itemize}

\paragraph{Small-step (Structural)}
\begin{itemize}
\item {\scriptsize{}(S-LEFT)} $\begin{array}{c}
E_{1}\rightarrow E_{1}'\\
\hline E_{1}+E_{2}\rightarrow E_{1}'+E_{2}
\end{array}$.
\item {\scriptsize{}(S-RIGHT)} $\begin{array}{c}
E\rightarrow E'\\
\hline n+E\rightarrow n+E'
\end{array}$.
\item {\scriptsize{}(S-ADD)} $\begin{array}{c}
\\
\hline n_{1}+n_{2}\rightarrow n_{3}
\end{array}$ (where $n_{3}=n_{1}+n_{2}$).
\item \emph{Reflexie transitive closure}: $E\rightarrow^{*}E'$ if $E=E'$
or there is a finite sequence $E\rightarrow E_{1}\rightarrow E_{2}\dots\rightarrow E_{k}\rightarrow E'$.
\item \emph{Normal form}: $E$ is in normal form (irreducable) if there
is no $E'$ s.t. $E\rightarrow E'$.
\end{itemize}

\subparagraph{Properties:}
\begin{itemize}
\item \emph{Determinacy}: For all $E_{1}$, $E_{2}$, if $E\rightarrow E_{1}$
and $E\rightarrow E_{2}$ then $E_{1}=E_{2}$.
\item \emph{Confluence}: For all $E$, $E_{1}$, $E_{2}$, if $E\rightarrow^{*}E_{1}$
and $E\rightarrow^{*}E$ then there exists $E'$ s.t. $E_{1}\rightarrow^{*}E'$
and $E_{2}\rightarrow^{*}E'$.
\item \emph{Strong normalisation}: No infinite sequence of expressions $E_{1},E_{2},E_{3}$
such that for all $i$, $E_{i}\rightarrow E_{i+1}$.
\end{itemize}

\paragraph{States}
\begin{itemize}
\item Partial function from variable numbers s.t. $s(x)$ is defined for
finitely many $x$. E.g. $s=\left(x\mapsto4,y\mapsto5,z\mapsto6\right)$.
\item \emph{Configuration} $\left\langle E,s\right\rangle $ means evaluate
$E$ w.r.t. state $s$.
\end{itemize}

\section{Register Machines}

\section{Turing Machines}

\section{Lambda Calculus}
\end{document}
